% $Id$

\documentclass[a4paper,12pt,twoside,titlepage]{book}

\usepackage{scaladoc}
\usepackage{fleqn}
\usepackage{modefs}
\usepackage{math}
\usepackage{scaladefs}
\renewcommand{\todo}[1]{}



\ifpdf
    \pdfinfo {
        /Author   (Martin Odersky)
        /Title    (Scala by Example)
        /Keywords (Scala)
        /Subject  ()
        /Creator  (TeX)
        /Producer (PDFLaTeX)
    }
\fi

\renewcommand{\doctitle}{Scala By Example\\[33mm]\ }
\renewcommand{\docauthor}{Martin Odersky\\[53mm]\ }

\begin{document}

\frontmatter
\makedoctitle
\clearemptydoublepage
\tableofcontents
\mainmatter
\sloppy

\input{ExamplesPart}
\bibliographystyle{alpha}
\bibliography{Scala}

\end{document}



\chapter{Implementing Abstract Types: Search Trees}

This chapter presents unbalanced binary search trees, implemented in
three different styles: algebraic, object-oriented, and imperative.
In each case, a search tree package is seen as an implementation
of a class {\em MapStruct}.
\begin{lstlisting}
  trait MapStruct[kt, vt] {
    trait Map with Function1[kt, vt] {
      def extend(key: kt, value: vt): Map;
      def remove(key: kt): Map;
      def domain: Stream[kt];
      def range: Stream[vt];
    }
    type map <: Map;
    val empty: map;
  }
\end{lstlisting}
The \code{MapStruct} class is parameterized with a type of keys
\code{kt} and a type of values \code{vt}. It
specifies an abstract type \code{Map} and an abstract value
\code{empty}, which represents empty maps.  Every implementation
\code{Map} needs to conform to that abstract type, which
extends the function type \code{kt => vt}
with four new
methods. The method \code{domain} yields a stream that enumerates the
map's domain, i.e. the set of keys that are mapped to non-null values.
The method \code{range} yields a stream that enumerates the function's
range, i.e.\ the values obtained by applying the function to arguments
in its domain.  The method
\code{extend} extends the map with a given key/value binding, whereas
\code{remove} removes a given key from the map's domain. Both
methods yield a new map value as result, which has the same
representation as the receiver object.

\begin{figure}[t]
\begin{lstlisting}
class AlgBinTree[kt extends Ord, vt] extends MapStruct[kt, vt] {
  private case
    Empty extends Map,
    Node(key: kt, value: vt, l: Map, r: Map) extends Map

  final class Map extends kt => vt {
    def apply(key: kt): vt = this match {
      case Empty => null
      case Node(k, v, l, r) =>
        if (key < k) l.apply(key)
        else if (key > k) r.apply(key)
        else v
    }

    def extend(key: kt, value: vt): Map = this match {
      case Empty => Node(k, v, Empty, Empty)
      case Node(k, v, l, r) =>
        if (key < k) Node(k, v, l.extend(key, value), r)
        else if (key > k) Node(k, v, l, r.extend(key, value))
        else Node(k, value, l, r)
    }

    def remove(key: kt): Map = this match {
      case Empty => Empty
      case Node(k, v, l, r) =>
        if (key < k) Node(k, v, l.remove(key), r)
        else if (key > k) Node(k, v, l, r.remove(key))
        else if (l == Empty) r
        else if (r == Empty) l
        else {
          val midKey = r.domain.head
          Node(midKey, r.apply(midKey), l, r.remove(midKey))
        }
    }

    def domain: Stream[kt] = this match {
      case Empty => []
      case Node(k, v, l, r) => Stream.concat([l.domain, [k], r.domain])
    }
    def range: Stream[vt] = this match {
      case Empty => []
      case Node(k, v, l, r) => Stream.concat([l.range, [v], r.range])
    }
  }
  def empty: Map = Empty
}
\end{lstlisting}
\caption{\label{fig:algbintree}Algebraic implementation of binary
search trees}
\end{figure}
We now present three implementations of \code{Map}, which are all
based on binary search trees. The \code{apply} method of a map is
implemented in each case by the usual search function over binary
trees, which compares a given key with the key stored in the topmost
tree node, and depending on the result of the comparison, searches the
left or the right hand sub-tree. The type of keys must implement the
\code{Ord} class, which contains comparison methods
(see Chapter~\ref{chap:classes} for a definition of class \code{Ord}).

The first implementation, \code{AlgBinTree}, is given in
Figure~\ref{fig:algbintree}. It represents a map with a
data type \code{Map} with two cases, \code{Empty} and \code{Node}.

Every method of \code{AlgBinTree[kt, vt].Map} performs a pattern
match on the value of
\code{this} using the \code{match} method which is defined as postfix
function application in class \code{Object} (\sref{sec:class-object}).

The functions \code{domain} and \code{range} return their results as
lazily constructed lists. The \code{Stream} class is an alias of
\code{List} which should be used to indicate the fact that its values
are constructed lazily.

\begin{figure}[thb]
\begin{lstlisting}
class OOBinTree[kt extends Ord, vt] extends MapStruct[kt, vt] {
  abstract class Map extends kt => vt {
    def apply(key: kt): v
    def extend(key: kt, value: vt): Map
    def remove(key: kt): Map
    def domain: Stream[kt]
    def range: Stream[vt]
  }
  module empty extends Map {
    def apply(key: kt) = null
    def extend(key: kt, value: vt) = Node(key, value, empty, empty)
    def remove(key: kt) = empty
    def domain = []
    def range = []
  }
  private class Node(k: kt, v: vt, l: Map, r: Map) extends Map {
    def apply(key: kt): vt =
      if (key < k) l.apply(key)
      else if (key > k) r.apply(key)
      else v
    def extend(key: kt, value: vt): Map =
      if (key < k) Node(k, v, l.extend(key, value), r)
      else if (key > k) Node(k, v, l, r.extend(key, value))
      else Node(k, value, l, r)
    def remove(key: kt): Map =
      if (key < k) Node(k, v, l.remove(key), r)
      else if (key > k) Node(k, v, l, r.remove(key))
      else if (l == empty) r
      else if (r == empty) l
      else {
        val midKey = r.domain.head
        Node(midKey, r(midKey), l, r.remove(midKey))
      }
    def domain: Stream[kt] = Stream.concat([l.domain, [k], r.domain])
    def range: Stream[vt] = Stream.concat([l.range, [v], r.range])
  }
}
\end{lstlisting}
\caption{\label{fig:oobintree}Object-oriented implementation of binary
search trees}
\end{figure}

The second implementation of maps is given in
Figure~\ref{fig:oobintree}.  Class \code{OOBinTree} implements the
type \code{Map} with a module \code{empty} and a class
\code{Node}, which define the behavior of empty and non-empty trees,
respectively.

Note the different nesting structure of \code{AlgBinTree} and
\code{OOBinTree}. In the former, all methods form part of the base
class \code{Map}. The different behavior of empty and non-empty trees
is expressed using a pattern match on the tree itself. In the
latter, each subclass of \code{Map} defines its own set of
methods, which override the methods in the base class. The pattern
matches of the algebraic implementation have been replaced by the
dynamic binding that comes with method overriding.

Which of the two schemes is preferable depends to a large degree on
which extensions of the type are anticipated. If the type is later
extended with a new alternative, it is best to keep methods in each
alternative, the way it was done in \code{OOBinTree}.  On the other
hand, if the type is extended with additional methods, then it is
preferable to keep only one implementation of methods and to rely on
pattern matching, since this way existing subclasses need not be
modified.

\begin{figure}
\begin{lstlisting}
class MutBinTree[kt extends Ord, vt] extends MapStruct[kt, vt] {
  class Map(key: kt, value: vt) extends kt => vt {
    val k = key
    var v = value
    var l = empty, r = empty

    def apply(key: kt): vt =
      if (this eq empty) null
      else if (key < k) l.apply(key)
      else if (key > k) r.apply(key)
      else v

    def extend(key: kt, value: vt): Map =
      if (this eq empty) Map(key, value)
      else {
        if (key < k) l = l.extend(key, value)
        else if (key > k) r = r.extend(key, value)
        else v = value
        this
      }

    def remove(key: kt): Map =
      if (this eq empty) this
      else if (key < k) { l = l.remove(key) ; this }
      else if (key > k) { r = r.remove(key) ; this }
      else if (l eq empty) r
      else if (r eq empty) l
      else {
        var mid = r
        while (!(mid.l eq empty)) { mid = mid.l }
        mid.r = r.remove(mid.k)
        mid.l = l
        mid
      }

    def domain: Stream[kt] = Stream.concat([l.domain, [k], r.domain])
    def range: Stream[vt] = Stream.concat([l.range, [v], r.range])
  }
  let empty = new Map(null, null)
}
\end{lstlisting}
\caption{\label{fig:impbintree}Side-effecting implementation of binary
search trees}
\end{figure}

The two versions of binary trees presented so far are {\em
persistent}, in the sense that maps are values that cannot be changed
by side effects. By contrast, in the next implementation of binary
trees, the implementations of \code{extend} and
\code{remove} do have an effect on the state of their receiver
object. This corresponds to the way binary trees are usually
implemented in imperative languages. The new implementation can lead
to some savings in computing time and memory allocation, but care is
required not to use the original tree after it has been modified by a
side-effecting operation.

In this implementation, \code{value}, \code{l} and \code{r} are
variables that can be affected by method calls.  The
class \code{MutBinTree[kt, vt].Map} takes two instance parameters
which define the \code{key} value and the initial value of the
\code{value} variable. Empty trees are represented by a
value \code{empty}, which has \code{null} (signifying undefined) in
both its key and value fields. Note that this value needs to be
defined lazily using \code{let} since its definition involves the
creation of a
\code{Map} object,
which accesses \code{empty} recursively as part of its initialization.
All methods test first whether the current tree is empty using the
reference equality operator \code{eq} (\sref{sec:class-object}).

As a program using the \code{MapStruct} abstraction, consider a function
which creates a map from strings to integers and then applies it to a
key string:
\begin{lstlisting}
def mapTest(def mapImpl: MapStruct[String, int]): int = {
  val map: mapImpl.Map = mapImpl.empty.extend("ab", 1).extend("bx", 3)
  val x = map("ab")             // returns 1
}
\end{lstlisting}
The function is parameterized with the particular implementation of
\code{MapStruct}. It can be applied to any one of the three implementations
described above. E.g.:
\begin{lstlisting}
mapTest(AlgBinTree[String, int])
mapTest(OOBinTree[String, int])
mapTest(MutBinTree[String, int])
\end{lstlisting}
}
 




\paragraph{Mixin Composition}
We can now define a class of \code{Rational} numbers that
support comparison operators.
\begin{lstlisting}
final class OrderedRational(n: int, d: int)
 extends Rational(n, d) with Ord {
  override def ==(that: OrderedRational) =
    numer == that.numer && denom == that.denom;
  def <(that: OrderedRational): boolean =
    numer * that.denom < that.numer * denom;
}
\end{lstlisting}
Class \code{OrderedRational} redefines method \code{==}, which is
defined as reference equality in class \code{Object}. It also
implements the abstract method \code{<} from class \code{Ord}.  In
addition, it inherits all members of class \code{Rational} and all
non-abstract members of class \code{Ord}. The implementations of
\code{==} and \code{<} replace the definition of \code{==} in class
\code{Object} and the abstract declaration of \code{<} in class
\code{Ord}. The other inherited comparison methods then refer to this
implementation in their body.

The clause ``\code{Rational(d, d) with Ord}'' is an instance of {\em
mixin composition}. It describes a template for an object that is
compatible with both \code{Rational} and \code{Ord} and that contains
all members of either class. \code{Rational} is called the {\em
superclass} of \code{OrderedRational} while \code{Ord} is called a
{\em mixin class}. The type of this template is the {\em compound
type} ``\code{Rational with Ord}''.

On the surface, mixin composition looks much like multiple
inheritance. The difference between the two becomes apparent if we
look at superclasses of inherited classes. With multiple inheritance,
both \code{Rational} and \code{Ord} would contribute a superclass
\code{Object} to the template. We therefore have to answer some
tricky questions, such as whether members of \code{Object} are present
once or twice and whether the initializer of \code{Object} is called
once or twice. Mixin composition avoids these complications.  In the
mixin composition \code{Rational with Ord}, class
\code{Rational} is treated as actual superclass of class \code{Ord}.
A mixin composition \code{C with M} is well-formed as long as the
first operand \code{C} conforms to the declared superclass of the
second operand \code{M}. This holds in our example, because
\code{Rational} conforms to \code{Object}. In a sense, mixin composition
amounts to overriding the superclass of a class.

\paragraph{Final Classes}
Note that class \code{OrderedRational} was defined
\code{final}. This means that no classes extending \code{OrderedRational}
may be defined in other parts of the program.
%Within final classes the
%type \code{this} is an alias of the defined class itself. Therefore,
%we could define our \code{<} method with an argument of type
%\code{OrderedRational} as a well-formed implementation of the abstract class
%\code{less(that: this)} in class \code{Ord}.

\chapter{\label{sec:simple-examples}Pattern Matching}

\todo{Complete}

Consider binary trees whose leafs contain integer arguments. This can
be described by a class for trees, with subclasses for leafs and
branch nodes:
\begin{lstlisting}
abstract class Tree;
case class Branch(left: Tree, right: Tree) extends Tree;
case class Leaf(x: int) extends Tree;
\end{lstlisting}
Note that the class \code{Tree} is not followed by an extends
clause or a body. This defines \code{Tree} to be an empty
subclass of \code{Object}, as if we had written
\begin{lstlisting}
class Tree extends Object {}
\end{lstlisting}
Note also that the two subclasses of \code{Tree} have a \code{case}
modifier.  That modifier has two effects. First, it lets us construct
values of a case class by simply calling the constructor, without
needing a preceding \code{new}. Example:
\begin{lstlisting}
val tree1 = Branch(Branch(Leaf(1), Leaf(2)), Branch(Leaf(3), Leaf(4)))
\end{lstlisting}
Second, it lets us use constructors for these classes in patterns, as
is illustrated in the following example.
\begin{lstlisting}
def sumLeaves(t: Tree): int = t match {
  case Branch(l, r) => sumLeaves(l) + sumLeaves(r)
  case Leaf(x) => x
}
\end{lstlisting}
The function \code{sumLeaves} sums up all the integer values in the
leaves of a given tree \code{t}. It is is implemented by calling the
\code{match} method of \code{t} with a {\em choice expression} as
argument (\code{match} is a predefined method in class \code{Object}).
The choice expression consists of two cases which botrelate a pattern with an expression. The pattern of the first case,
\code{Branch(l, r)} matches all instances of class \code{Branch}
and binds the {\em pattern variables} \code{l} and \code{r} to the
constructor arguments, i.e.\ the left and right subtrees of the
branch.  Pattern variables always start with a lower case letter; to
avoid ambiguities, constructors in patterns should start with an upper
case letter.

The effect of the choice expression is to select the first alternative
whose pattern matches the given select value, and to evaluate the body
of this alternative in a context where pattern variables are bound to
corresponding parts of the selector. For instance, the application
\code{sumLeaves(tree1)} would select the first alternative with the
\code{Branch(l,r)} pattern, and would evaluate the expression
\code{sumLeaves(l) + sumLeaves(r)} with bindings
\begin{lstlisting}
l = Branch(Leaf(1), Leaf(2)), r = Branch(Leaf(3), Leaf(4)).
\end{lstlisting}
As another example, consider the following class
\begin{lstlisting}
abstract final class Option[+a];
case object None extends Option[All];
case class Some[a](item: a) extends Option[a];
\end{lstlisting}
...

%\todo{Several simple and intermediate examples needed}.

\begin{lstlisting}
def find[a,b](it: Iterator[(a, b)], x: a): Option[b] = {
  var result: Option[b] = None;
  while (it.hasNext && result == None) {
    val (x1, y) = it.next;
    if (x == x1) result = Some(y)
  }
  result
}
find(xs, x) match {
  case Some(y) => System.out.println(y)
  case None => System.out.println("no match")
}
\end{lstlisting}

\comment{


class MaxCounter {
  var maxVal: Option[int] = None;
  def set(x: int) = maxVal match {
    case None => maxVal = Some(x)
    case Some(y) => maxVal = Some(Math.max(x, y))
  }
}
\end{lstlisting}
}
\comment{
\begin{lstlisting}
class Stream[a] = List[a]

module Stream {
  def concat(xss: Stream[Stream[a]]): Stream[a] = {
    let result: Stream[a] = xss match {
      case [] => []
      case [] :: xss1 => concat(xss1)
      case (x :: xs) :: xss1 => x :: concat(xs :: xss1)
    }
    result
  }
}
\end{lstlisting}
}
\comment{
}
%\todo{We also need some XML examples.}

\end{document}



  case ([], _) => ys
  case (_, []) => xs
  case (x :: xs1, y :: ys1) =>
    if (x < y) x :: merge(xs1, ys) else y :: merge(xs, ys1)
}

def split (xs: List[a]): (List[a], List[a]) = xs match {
  case [] => ([], [])
  case [x] => (x, [])
  case y :: z :: xs1 => val (ys, zs) = split(xs1) ; (y :: ys, z :: zs)
}

def sort(xs: List[a]): List[a] = {
  val (ys, zs) = split(xs)
  merge(sort(ys), sort(zs))
}


def sort(a:Array[String]): Array[String] = {
  val pivot = a(a.length / 2)
  sort(a.filter(x => x < pivot)) ++
       a.filter(x => x == pivot) ++
  sort(a.filter(x => x > pivot))
}

def sort(a:Array[String]): Array[String] = {

  def swap (i: int, j: int): unit = {
    val t = a(i) ; a(i) = a(j) ; a(j) = t
  }

  def sort1(l: int, r: int): unit = {
    val pivot = a((l + r) / 2)
    var i = l, j = r
    while (i <= r) {
      while (i < r && a(i) < pivot) { i = i + 1 }
      while (j > l && a(j) > pivot) { j = j - 1 }
      if (i <= j) {
        swap(i, j)
        i = i + 1
        j = j - 1
      }
    }
    if (l < j) sort1(l, j)
    if (j < r) sort1(i, r)
  }

  sort1(0, a.length - 1)
}

class Array[a] {

  def copy(to: Array[a], src: int, dst: int, len: int): unit
  val length: int
  val apply(i: int): a
  val update(i: int, x: a): unit

  def filter (p: a => boolean): Array[a] = {
    val temp = new Array[a](a.length)
    var i = 0, j = 0
    for (i < a.length, i = i + 1) {
      val x = a(i)
      if (p(x)) { temp(j) = x; j = j + 1 }
    }
    val res = new Array[a](j)
    temp.copy(res, 0, 0, j)
  }

  def ++ (that: Array[a]): Array[a] = {
    val a = new Array[a](this.length + that.length)
    this.copy(a, 0, 0, this.length)
    that.copy(a, 0, this.length, that.length)
  }

static

  def concat [a] (as: List[Array[a]]) = {
    val l = (as map (a => a.length)).sum
    val dst = new Array[a](l)
    var j = 0
    as forall {a => { a.copy(dst, j, a.length) ; j = j + a.length }}
    dst
  }

}

module ABT extends AlgBinTree[kt, vt]
ABT.Map
